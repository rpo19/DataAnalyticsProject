\section{Esplorazione dataset}\label{EsplorazioneDataset}
%TODO: dire in generale contesto e di cosa parla il dataset, formato dalle due tabelle descritte in seguito
Il dataset oggetto di studio in questo progetto è relativo ad una partizione di prodotti di Amazon e dalle rispettive recensioni rilasciate dagli utenti. Esso è composto dalle due tabelle \textit{product} e \textit{reviews} descritte nel seguito sezioni \ref{Prodotti} e \ref{Recensioni}.

\subsection{Prodotti}\label{Prodotti}
La tabella \textit{product} è quella su cui si baserà l'analisi di rete e si compone dei seguenti campi:
\begin{itemize}
    \item \textbf{\_id}: codice alfanumerico identificativo del prodotto
    \item \textbf{title}: titolo del prodotto; rappresenta già di per sè una breve descrizione delle sue caratteristiche
    \item \textbf{category}: categoria di appartenenza del prodotto
    \item \textbf{price}: prezzo del prodotto in euro
    \item \textbf{avg\_rating}: media delle stelle delle recensioni sul prodotto; valore con una cifra decimale nell'intervallo [1, 5] 
    \item \textbf{reviews\_number}: numero totale di recensioni ricevute dal prodotto
    \item \textbf{questions\_number}: numero totale di domande poste dagli utenti sul prodotto
    \item \textbf{pictures}: url delle immagini relative al prodotto
    \item \textbf{description}: descrizione dettagliata del prodotto
    \item \textbf{features}: caratteristiche del prodotto
    \item \textbf{versions}: lista degli identificativi dei prodotti che rappresentano altre versioni dello stesso
    \item \textbf{bought\_together}: lista degli identificativi dei prodotti che vengono più spesso acquistati con il prodotto in questione nello stesso ordine; rappresenta una relazione tra prodotti
    \item \textbf{also\_bought}: lista degli identificativi dei prodotti che vengono più spesso acquistati in combinazione con il prodotto in questione; non necessariamente l'acquisto avviene nello stesso ordine; rappresenta una relazione tra prodotti
    \item \textbf{also\_viewed}: lista degli identificativi dei prodotti che vengono spesso più visualizzati da chi visualizza il prodotto in questione; rappresenta una relazione tra prodotti
\end{itemize}
Questa tabella raccoglie un totale di 20,459 prodotti, i quali si dividono in 37 categorie. La distribuzione nelle categorie è mostrata in figura \ref{fig:distribuzioneCategorie} e si può come essa sia particolarmente sbilanciata. \\
Le relazioni \textit{bought\_together}, \textit{also\_bought} e \textit{also\_viewed} che legano i prodotti sono fondamentali per effettuare un'analisi di rete e saranno approfondite nella sezione \ref{ReteProdotti}.
%TODO: mettere figura distribuzioneCategorie

\subsection{Recensioni}\label{Recensioni}
La tabella \textit{reviews}, che sarà oggetto di una Sentiment Analysis, si compone dai seguenti campi:
\begin{itemize}
    \item \textbf{\_id}: codice alfanumerico identificativo della recensione
    \item \textbf{product}:codice alfanumerico identificativo del prodotto a cui si riferisce la recensione
    \item \textbf{title}: titolo della recensione
    \item \textbf{author-id}: codice alfanumerico identificativo dell'autore della recensione
    \item \textbf{author-name}: nome dell'autore della recensione
    \item \textbf{date}: data in cui è stata scritta la recensione
    \item \textbf{rating}: voto in stelle dato al prodotto in questione; valore intero nell'intervallo [1, 5]
    \item \textbf{helpful}: numero di volte in cui la recensione è stata segnalata come utile da altri utenti
    \item \textbf{verified}: flag che rappresenta se la recensione sia relativa ad un acquisto verificato
    \item \textbf{body}: testo della recensione
\end{itemize} 
La tabella raccoglie un totale di 1,988,854 recensioni. In figura \ref{fig:distribuzioneRating} è possibile osservare una distribuzione fortemente sbilanciata verso le valutazioni positive, per questo motivo sarà importante bilanciare i dati nella fase di Sentiment Analysis. Come verrà mostrato alla sezione \ref{SentimentAnalysis}, gli attributi \textit{helpful} e \textit{verified} assumeranno un ruolo importante nel filtraggio del numero di recensioni. L'attributo \textit{body} costituirà invece il testo oggetto dell'analisi.

%TODO: mettere figura distribuzioneRating

%TODO: magari mettere le word cloud raggruppate per stelle... non so se per ogni stella o aggregare 12 e 45 anche qua (fare l'intersezione che diceva gabri)