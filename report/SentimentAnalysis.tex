\section{Sentiment Analysis recensioni}\label{SentimentAnalysis}
%TODO: dire subito bene gli obiettivi e cosa sarà fatto in generale
%TODO: dire che consideriamo positive 4-5 e negative 1-2... lo direi già qui perchè poi serve in entrambi gli approcci

\subsection{Approccio Lexicon Based}
%TODO: dire brevemente cos'è sentix, ha diversi score, restituisce polarità [-1,1]...  mettiamo il riferimento e bona

%TODO: testato su campione di 1000 positive e 1000 negative con scarsi risultati... quasi sempre valori positivi
    % mostrare accuratezza e magari l'andamento rating vs polarity su un paio di prodotti

%TODO: fa schifo -> non adatto per capire utilizzo medio delle 3 stelle (già descritto sopra come obiettivo...)

\subsection{Approccio Machine Learning}
%TODO: dire che si considera il binario, GABRI the machine learner se ha in mente premesse da fare le faccia

%TODO: dire che avendo a disposizione un grande numero di recensioni decidiamo che è possiamo usare solo quelle filtrate verified e helpful > 1 per addestrare bene e in modo piu AFFIDABILE il modello

%AAAAAAAAAAAAAAAA: da qui in poi i risultati che mostriamo dovremmo tirarli fuori dalle recensioni non filtrate
%TODO: confrontare andamento raggruppato per mese di qualche prodotto significativo (significativo = ha tante recensioni, quindi fa un grafico piu realistico) -> molto simili...
%TODO: calcolare un eventuale coefficiente di correlazioneee?

%TODO: mostrare e discutere risultati ottenuti effettuando previsione sulle 3 stelle

\subsection{Individuazione topics}
%TODO:  Montamagno king of LDA