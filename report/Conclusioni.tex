\section{Conclusioni}\label{Conclusioni}
In questa analisi sono stati affrontati aspetti legati alla rete dei prodotti e al sentimento delle recensioni. 
\\\\
Per quanto riguarda la rete dei prodotti, essa è stata costruita sulla base delle relazioni di acquisto che li lega. Effettuandone l'analisi, come prima cosa è stato possibile identificare i prodotti che più spesso compaiono in combinazione con l'acquisto di altri beni. Inoltre sono stati individuati i prodotti che garantirebbero una migliore raggiungibilità partendo da altri acquisti e che quindi hanno una maggiore visibilità all'interno della rete. \\
È stata poi rilevata la presenza di communities formate da gruppi di prodotti spesso acquistati in combinazione. Ciò ha permesso quindi di mostrare come una divisione guidata dagli acquisti sia più significativa rispetto al semplice raggruppamento per categorie. Per ogni community sono stati individuati i prodotti più rilevanti e fornite le sfere d'interesse. 
\\\\
Passando alla parte relativa alla valutazione del sentimento delle recensioni è stato riscontrato che un approccio basato sul lessico sia risultato poco efficace. Ha raggiunto risultati migliori, invece, un approccio di machine learning supervisionato che utilizza il modello della regressione logistica, il quale è stato utilizzato per raggiungere gli obiettivi preposti. Tramite questo modello è stato possibile fornire l'andamento del sentimento nel tempo per specifici prodotti, notando una forte coerenza con l'andamento del numero di stelle. Inoltre, è stato dimostrato che l'utilizzo medio delle 3 stelle da parte degli utenti non avesse tendenze verso una specifica polarità. 
\\\\
